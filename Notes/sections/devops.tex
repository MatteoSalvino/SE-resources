\section{DevOps}
In simple words DevOps means a combination of software development operations. It's an approach wide spread in modern software engineering, in which the operations started to make use many of the same techniques as developers for their systems work. The business problems that emerged during the last year are the following :
\begin{itemize}
    \item need more time to respond to market changes
    \item deployments held off to avoid risk
    \item slow and error prone releases
    \item fix and maintain rather than innovate
    \item unstable operations as fixes take more time
    \item IT is frequently seen as the bottleneck in the transition of "concept to cash".
\end{itemize}
The symptoms of these problems are that the developers tends always to work on their machine in order to minimizing the risk that the software release is unstable on another platform, is needed to have production environment access to diagnose issues, servers are not available for deployment (it could fails due to incorrect configuration), fix performed after a specific day, releases can have a lot of fails. The business trends that moves to DevOps are frequent deployments, faster recovery from failures, decreasing of failure rate, shorter lead times and to reach a better customer satisfaction. If DevOps is correctly implemented, it increases the velocity, reduce the downtime and human errors. DevOps is based on the following principles :
\begin{itemize}
    \item in big companies there are teams that are in charge of doing the development of the application. Typically, they change, modify and test the software. When the application is ready, they move it to IT operations stuff. They represent the team in charge of maintain the operations really run on the infrastructure, monitor the server infrastructure performance and this implies that they have to enhance the stability and the maintenance of the service. The fundamental aspect is the quality of this process. We are moving from the single person responsibility to a collective responsibility, shared understanding and service delivery.
    \item DevOps is a set of practices that emphasize the collaboration and communication of both software developers and IT professionals while automating the process of software delivery and infrastructure changes. Its goal is to bridge the gap between agile software development and operations. This is possible by unifying people, process and products in order to have continuous delivery of value to our end users.
    \item the basic idea is that the code should be managed in a code repository, i.e. a remote repository accessible by all developers and people involved in the process. Then there should be an automatic building process and testing process. Then we have the deployment process managed in a structured way (e.g. through a database), and finally it will be monitored and potentially improved.
\end{itemize}
The current software development situation is constituted by well separated areas : business, development and operations. In this case, each area doesn't communicate too much with the other ones. The idea with DevOps is that the business moves and we have a unique team that manages the development and the operations. The \textbf{continuous integration} is a fundamental aspect in DevOps, since it's the process of integrating code into a mainline code base. Its main key elements are : version/source control, frequent commits, build automation as well the testing, test outcome results availability, code and build stability, code quality and coverage. In this ways the bugs are detected very early, we have an immediate feedback on system-wide impact of local changes, we have a constant availability of the current build for testing, demo or release purposes, enforces discipline of frequent automated testing and allow to have faster time to release with repeatable process. However, it has also some disadvantages such as : automated test suites requires considerable amount of work to setup and also for ongoing needs, work involved to setup a build system, value added depends on the quality of tests and how testable the code really is, build queueing up can slow down everyone and partial code could easily be pushed and therefore integration tests could fail until the feature is complete. The \textbf{continuous delivery and deployment} are other important aspects in DevOps. The first one means to make sure that your software is always production ready throughout its entire life cycle, i.e. any build could potentially be released to users at the touch of a button using a fully automated process in a matter of seconds or minutes. The continuous deployment is the practice of releasing every good build to users. It deploy every change that passes the automated tests to production and is the next phase of continuous delivery.\\\\There are $9$ types of DevOps tools which has known before choosing for the project :
\begin{itemize}
    \item \textbf{Collaboration tools} :this type of tool is crucial to helping teams works together more easily, regardless of time zones or location. It's a rapid action oriented communication designed to share knowledge and save time
    \item \textbf{Planning tools} : this type of tool is designed to provide transparency to stakeholder and participants.
    \item \textbf{Source control tools} : tools of this sort make up the building blocks for the entire process ranging across all key assets.
    \item \textbf{Issue tracking tools} : these tools increase responsiveness and visibility.
    \item \textbf{Configuration management tools} : without this type of tool, it would impossible to enforce desired state norms or achieve any sort of consistency at scale.
    \item \textbf{Database DevOps tools} : the database needs to be an honored member of the managed resources family.
    \item \textbf{Continuous integration tools} : this type of tool provide an immediate feedback loop by regularly merging code.
          \textbf{Automated testing tools} :  tools of this sort are tasked with verifying code quality before passing the build.
    \item \textbf{Deployment tools} : these tools are essential to checking those boxes.
\end{itemize}