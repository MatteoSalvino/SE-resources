\section{Scrum}
Scrum is an agile process that allows us to focus on delivering the highest business value in the shortest time. It allows us to rapidly and repeatedly inspect actual working software (every two week to one month). The business set the priorities. Team self-organize to determine the best way to deliver the highest priority features. Every two weeks to a month anyone can see real working software and decide to release it as is or continue to enhance it for another sprint. The requirements of the customer are captured as items in a list of "product backlog". The core values of Scrum are the following :
\begin{itemize}
    \item \textbf{Commitment} : teams commit to their goals for the sprint, product owners commit to ordering the product backlog and ScrumMasters commit to removing any obstacles along the way in order to simplify the flow of product development. The Scrum team should do whatever is necessary in order to meet their goals, and it's important that they are empowered to do so.
    \item \textbf{Focus} : for a team to be able to complete its work, its members must be allowed to focus. The ScrumMaster doesn't allow changes in the sprint's commitment so that the team may keep its focus. When a team gives its full attention to the problem, its work is much more productive, predictable and fulfilling.
    \item \textbf{Openness} : as Scrum uses empirical process control to make progress through a project, it's essential that the results and experience of an experiment (i.e. a sprint) are visible. Once visibility exists, inspection and adaption can occur.
    \item \textbf{Respect} : in order to be its best, a team's members need to respect for each other and, the knowledge that each brings to the table, experiences, working styles and personalities. Respect doesn't come for free, it's earned. Scrum team members should be dedicated, cross-functional, empowered and self-organizing.
    \item \textbf{Courage} : it takes buckets of courage for a ScrumMaster to apply Scrum the way it was intended. One of the primary responsibilities of the ScrumMaster is to help the organization identify its weaknesses so that it may improve. This takes courage. Sometimes, a team has to push back on the product owner when asked to take on too much during a sprint. It takes courage to say no to that sort of pressure. A product owner must have courage when communicating with other stakeholders about the reality of a project.
\end{itemize}
\subsection{Roles}
\paragraph{Scrum team} The Scrum team includes the product owner, ScrumMaster and the team members, whereas the Scrum delivery team is a subset made of only the technical team members. The whole Scrum team huddles around a problem (i.e. a requirement from the Product Backlog) and innovates solutions. Scrum teams should be five to nine team members, dedicated to the life of the project, cross-functional, empowered and self-organizing. Scrum teams plan, estimate and commit to their work, rather than a manager performing these activities for them. The end goal of the team is to deliver a potentially shippable product increment that meets an agreed-upon Definition of Done each and every sprint.
\paragraph{Product owner} The product owner is responsible for the product's success. In other words, while the team is responsible for delivering a quality solution, the product owner is responsible for knowing his market and user needs well enough to guide the team towards a marketable release sprint after sprint. In a project there should be one and only one product owner who makes final decisions about the direction of the product and the order in which features should be developed. The product owner, since he is represent the "what" and "why" of the system, should be available to the team to have regular dialog about the requirements in the product backlog; additionally, the product owner must make the product vision clear to everyone on the team and regularly maintain the product backlog in keeping with the product vision. The product owner always keeps the next set of product backlog items in a ready state so that the team always has work in the queue for the next sprint.
\paragraph{ScrumMaster} The ScrumMaster safeguards the process. He/she understands the reasons behind and for an empirical process, and does his or her best to keep product development flowing as smoothly as possible. This leader protects team members from interruptions in order to keep them focused on their sprint commitments. The ScrumMaster also facilitate all Scrum meetings, ensuring that everyone on the team understands the goals and that they share a commitment together as a true team and not just as a collection of individuals. She/he removes obstacles that prevent a steady flow of high-value features.
\subsection{Ceremonies}
A sprint is an iteration defined by a fixed start and end data; it's kicked off by sprint planning and concluded by the sprint review and retrospective. The team meets daily, in a daily scrum meeting, to make their work visible to each other and synchronize based on what they've learned. Lets discuss in details these phases one at a time.
\paragraph{Sprint planning} During sprint planning the product owner and the team discuss the highest priority items in the product backlog and brainstorm a plan to implement those items. The set of chosen product backlog items and their subsequent tasks collectively is referred to as the team's sprint backlog. The sprint planning meeting is time-boxed to eight hours for a 30-day sprint, reduced proportionally for shorter sprints. The meeting is constituted by two parts : the first one is driven by the product owner who presents the most important product backlog items (with the support of drawings, mockups, etc.) and clarifies question from the development team about what he/she wants and why he/she wants it. The second part is driven by the Scrum delivery team who work together to brainstorm approach and eventually agree on a plan. It's at the start of this second part that the sprint actually begins. Of course, teams are always searching for ways to make planning faster and more efficient. The result of sprint planning is a sprint backlog that is comprised of selected product backlog items for the sprint, along with the correlating tasks identified by the team in the second part of sprint planning.
\paragraph{Sprint review} The sprint review provides the opportunity for stakeholders to give feedback about the emerging product in a collaborative setting. In this meeting, the team, product owner, ScrumMaster and any interested stakeholders meet to review and talk about how the product is shaping up, which features may need to change and perhaps discuss new ideas to add to the product backlog. It's common for a ScrumMaster to summarize the event of the sprint, any major obstacles that the team ran into, and so on, and of course the team should always demo what they've accomplished by the sprint's end. This meeting is time-boxed to four hours for a 30-day sprint.
\paragraph{Sprint retrospective} During the final spring meeting, the sprint retrospective, team members discuss events of the sprint, identify what worked well for them, what didn't work so well and take on action items for any changes that they would like to make for the next sprint. The ScrumMaster will take on any actions that the team doesn't feel it can handle. The ScrumMaster reports progress to the team regarding these obstacles in subsequent sprints. This meeting is time-boxed to three hours.
\paragraph{Daily scrum meeting} In the daily scrum meeting team members make their progress visible so that they can inspect and adapt toward meeting their goals. The meeting is held at the same time and in the same place, decided upon by the team. Even  though a team makes its best attempt at planning for a sprint, things can change in flight. In this 15-minute meeting, team members discuss what they did since yesterday's meeting, what they plan to do by tomorrow's meeting, and to mention any obstacles that may be in their way. The ScrumMaster record any obstacles that the team members feel they cannot fix for themselves and will attempt to remove them after the meeting. The scrum delivery team members, product owner and ScrumMaster are participants in the meeting. Anyone else is welcome to attend but only as observers.
\subsection{Artifacts}
\paragraph{Product backlog} The product backlog is the product owner's "wish list". Anything and everything that they think they might want in the product goes in this list. The product owner maintains the product backlog, although other stakeholders should have visibility of and the ability to suggest new items for the list. The product owner prioritizes the product backlog, listing the most important or most valuable items first. Once a team selects items for a sprint, those items and their priorities are locked; however, priorities and details for any not-started work may change at any time. Through this mechanism, teams are able to focus on this sprint's work while the product owner retains maximum flexibility in ordering the next sprint's work.
\paragraph{Sprint backlog} Owned by the team, the sprint backlog reflects the product backlog items that the team committed to in sprint planning, as well as the subsequent tasks and reminders. Team members update it every day to reflect how many hours remain on his/her task; team members may also remove tasks, add tasks or change tasks as the sprint is started.
\paragraph{Product increment} The product increment is a set of features, user stories or other deliverables completed by the team in the sprint. It should be potentially shippable (i.e. it must have enough quality to give it to the users). The product owner is responsible for accepting the product increment during each sprint, according to the agreed-upon Definition of Done and acceptance criteria for each sprint deliverable. Without a product increment, the product owner and other stakeholders have no way to inspect and adapt the product. A team must keep its progress visible at all times. It will create many additional artifacts in order to ensure visibility. Some common visibility tools are the release and sprint burndown charts. A burndown chart display of what work has been completed and what is left to complete. It is provided for each developer or work item. It's updated every day and make best guess about hours/points completed each day. A possible variation of this chart is called release burndown chart, which display how much work remain in the release backlog at the end each sprint (it shows overall progress).
