\section{User story}
BDD asks questions about the behavior of the application before and during development to reduce miscommunication.
The application's requirements are written as \textbf{user stories}.
A user story has the following structure:
\begin{itemize}
    \item \textbf{As a} [kind of stakeholder]
    \item \textbf{So that} [I can achieve some goal]
    \item \textbf{I want to} [do some task].
\end{itemize}
The idea is to use user stories as acceptance tests before the code is written.
A measure of team productivity could be the average number of stories/week ? We have a problem that some stories could be more difficult than others.
So, a simple fix is to assign to each user story a rate on a simple integer scale: for example, we could assign 1 for simple stories, $2$ for medium stories, and $3$ for very complex stories.
In this new setting, the velocity is given by the relationship between the average number of points and the week.
There are some guidelines to define properly a user story.
One of those is, if the rate assigned to a user story is greater or equal to $5$, then we could think to split this story into simpler stories.
Another guideline which each user story should follow, is the SMART approach:
\begin{itemize}
    \item \textbf{Specific and Measurable}: each scenario should be testable. It implies known good input and expected results exist. For example, Given some specific starting condition, When I take specific action X, Then one or more specific events should happen.
    \item \textbf{Achievable}: the ideal case is to complete user stories in one iteration. In the real world, this situation never occurs. If we can't deliver the feature in one iteration then we deliver a subset of stories.
    \item \textbf{Relevant}: of course, the user stories should represent an important feature. (see also 5 Whys approach). Otherwise, we can delete useless stories.
    \item \textbf{Timeboxed}: naturally if a story exceeds the time budget we need to stop it. We can divide it into simpler stories or reschedule what is left undone. This is an important characteristic, because in this way we avoid underestimating the length of the project.
\end{itemize}
When working with the customers it is useful to present user stories also in graphical form via Lo-Fi UI that gives to the customer a high-level idea of how the application will look.
Another important aspect is to include also a storyboard that shows how the UI changes based on user actions.
